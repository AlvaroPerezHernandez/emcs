\documentclass{article}

\usepackage[spanish]{babel}
\usepackage{amsmath}
\usepackage[utf8]{inputenc}
\usepackage[T1]{fontenc}

\title{metodosimplex}
\author{AlvaroJuan}

\begin{document}

\maketitle

\section{Introduccíon}
\label{sec:itroduccion}

EL metodo simplex es un algoritmo para resolver problemas de
programacón lineal.
Fue inventado por George Bernard Dantzig en el año 1947.

\section{Ejemplo}
\label{sec:ejemplo}

Resuelve el siguiente problema 
\begin{equation*}
 \begin{aligned}
\text{Maximizar} \quad & 2x_{1}+x_{2}\\
\text{sujeto a} \quad &
  \begin{aligned}
    x_{1}-x_{2}
    \leq 2\\
   -2x_{1}+x_{2} &\leq 2\\
   3x_{1}+4x_{2} &\leq 12\\
   x_{1}+x_{2} &\geq 1\\
    x_{1},x_{2} &\geq 0
  \end{aligned}
\end{aligned}
\end{equation*}

Como en una de las desigualdades aparecen las variables del lado
izquierdo de un simbolo $\geq$, multiplicamos ambos miembros de esa
desigualdad por $-1$ para obtener la forma estandar.

\begin{equation*}
 \begin{aligned}
\text{Maximizar} \quad & 2x_{1}+x_{2}\\
\text{sujeto a} \quad &
  \begin{aligned}
   x_{1}-x_{2}\leq 2\\
  -2x_{1}+x_{2} &\leq 2\\
   3x_{1}+4x_{2} &\leq 12\\
  -x_{1}-x_{2} &\leq -1\\
  x_{1},x_{2} &\geq 0
  \end{aligned}
\end{aligned}
\end{equation*}

Para obtener la forma simplex, añadimos una variable de holgura

\begin{equation*}
 \begin{aligned}
\text{Maximizar} \quad & 2x_{1}+x_{2}\\
\text{sujeto a} \quad &
  \begin{aligned}
   x_{1}-x_{2}+x_3 &= 2\\
   -2x_{1}+x_{2}+x_4 &= 2\\
   3x_{1}+4x_{2}+x_5 &= 12\\
  -x_{1}-x_{2}+x_6 &=-1\\
  x_{1},x_{2},x_3,x_4,x_5,x_6 &\geq 0
  \end{aligned}
\end{aligned}
\end{equation*}

a continuación obtenemos un \emph{tablero simplex} despejando las
variables de holgura.

\begin{equation*}
  \begin{aligned}
  x_3 &= 2x_1+x_2\\
  x_4 &=2+2x_1-x_2\\
  x_5 &=12-3x_1-2x_2\\
  x_6 &=-1+x_1+x_2\\
  \hline
  z &=\phantom{-1} +2x_1+x_2
  
\end{aligned}
\end{equation*}

\end{document}